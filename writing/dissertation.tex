\documentclass[a4paper,11pt]{article}  % font size 11pt set here
\usepackage{graphicx} % graphicx is the package for figures
\usepackage[top=2cm,bottom=2cm,left=2cm,right=2cm]{geometry} % make the margins the right size
\usepackage{parskip} % parskip separates non-indented paragraphs by a blank line.
\usepackage[style=ieee, backend=biber]{biblatex} % bibliography and citation handler
\addbibresource{dissertation.bib} % tell biblatex where our bibliography is

% Here's the paper's title and author details.
\title{Automated Toxicity in Brazilian Political Discourse}
\author{Kai Barber-Harris}
\date{}   % Remove this line if you actually want to see the date

% Here is the actual beginning of the document
\begin{document}
	
	%=========================================================================================================
	% Title page
	%=========================================================================================================
	\maketitle
	
	\begin{abstract}
		An abstract is a short statement about your paper designed to give the reader a complete, yet 
		concise, understanding of your paper's research and findings. It is a mini-version of your paper.
		A well-prepared abstract allows a reader to quickly and accurately identify the basic content of 
		your paper. Readers should be able to read your abstract to see if the related research is of 
		interest to them.  It is a bit like advertising: having read the abstract, do you want to pay to see the
		whole paper, which might be behind a paywall.
		Do not include citations in an abstract.  100-200 words.
	\end{abstract}
	
	\vfill
	
	\begin{center}
		I certify that all material in this dissertation which is not my own work has been identified.
	\end{center}
	
	\vspace{5mm}
	
	Signed: 
	
	\vspace{50mm}
	
	\newpage
	%=========================================================================================================
	% Content starts here
	%=========================================================================================================
	\section{Introduction}
	\label{sec:intro}
	
	%=========================================================================================================
	% Content ends here
	%=========================================================================================================
	
	\newpage
	\printbibliography
	
\end{document}